\documentclass[14pt,a4paper,oneside]{extbook}

\usepackage{indentfirst}
\usepackage{float}
\usepackage[T2A]{fontenc}
\usepackage[utf8]{inputenc}
\def\BabelCaseHack{}
\usepackage[ukrainian]{babel}
\usepackage[legalpaper, margin=2cm]{geometry}
\usepackage{graphicx}
\usepackage{fancyhdr}
\usepackage{titlesec}
\usepackage{enumitem}
\usepackage{tempora}
\usepackage{listings}
\usepackage{xcolor}
\usepackage{multirow}
\usepackage{setspace}
\usepackage{array}
\usepackage{tabularx}

\pagestyle{fancy}
\renewcommand{\headrulewidth}{0pt}
\lhead{}
\chead{}
\rhead{}
\lfoot{}
\cfoot{\thepage}
\rfoot{}
\setstretch{1.6}

\titleformat{\chapter}{\normalfont\bfseries\MakeUppercase}{}{14pt}{}
\titleformat*{\section}{\normalsize\bfseries}
\titleformat*{\subsection}{\normalsize\bfseries}
\titleformat*{\subsubsection}{\normalsize\bfseries}
\titleformat*{\paragraph}{\normalsize\bfseries}
\titleformat*{\subparagraph}{\normalsize\bfseries}
\titlespacing*{\chapter}{0pt}{0pt}{0pt}
\titlespacing*{\section}{0pt}{0pt}{0pt}
\titlespacing*{\subsection}{0pt}{0pt}{0pt}
\titlespacing*{\subsubsection}{0pt}{0pt}{0pt}
\titlespacing*{\paragraph}{0pt}{0pt}{0pt}
\titlespacing*{\subparagraph}{0pt}{0pt}{0pt}

\newcommand{\LabName}{База даних обліку успішності в ЗВО}
\newcommand{\LabTheme}{Основи системного аналізу}

\newcommand{\DefaultFigure}[3]{
	\begin{figure}[H]
		\centering
		\includegraphics[width=\textwidth]{img/#1}
		\caption{#2}
		\label{pic:#3}
	\end{figure}
}

\lstset {
    commentstyle=\color{olive},
    keywordstyle=\color{red},
    numberstyle=\tiny,
    stringstyle=\color{teal},
	basicstyle={\ttfamily \scriptsize},
    breaklines=true
}

\begin{document}
\begin{titlepage}

	\begin{center}
		
		\textbf{
			МІНІСТЕРСТВО ОСВІТИ І НАУКИ УКРАЇНИ\\
			НАЦІОНАЛЬНИЙ ТЕХНІЧНИЙ УНІВЕРСИТЕТ УКРАЇНИ\\
			«КИЇВСЬКИЙ ПОЛІТЕХНІЧНИЙ ІНСТИТУТ ІМ. ІГОРЯ СІКОРСЬКОГО»\\
			ІНСТИТУТ ПРИКЛАДНОГО СИСТЕМНОГО АНАЛІЗУ\\
		}
		
		\vspace{10cm}

		Розрахунково-графічна робота\\
		з дисципліни «\LabTheme »\\
		на тему «\LabName »
		
		\vspace{3cm}

		\begin{flushright}
			Виконали:\\
			Студенти групи ДА-01\\
			ННК «ІПСА»\\
			Андрушкой Андрій, \\
			Зарицький Кирило, \\
			Колядюк Владислав, \\
			Скуратовський Артем\\
			Варіант № 3
		\end{flushright}
		
	\vfill
	Київ — 2022

	\end{center}

\end{titlepage}
\setcounter{page}{1}
\tableofcontents
%Document starts here
\chapter{Вступ}
	В сучасних закладах вищої освіти існує проблема збереження даних студентів. Студенти зацікавлені в можливості зручного перегляду оцінок за дисципліни. Викладачі зацікавлені в можливості зручного введення цих дисциплін. Адміністрація ж потребує дані про діяльність як студентів, так і викладачів.
	
	Додамо до цього пандемію 2019 року, що спричинила масове поширення вірусу, що суттєво вплинуло на освітній процес. Було введене дистанційне навчання. В його умовах постала необхідність віддаленого контролю і перегляду успішності студентів.

	Відповідно для успішного співпрацювання студентів та робітників ЗВО потрібна система централізованого зберігання успішності студентів.

	У світі практично жодна автоматизована система управління не обходиться без баз даних. БД є основою більшості сучасних довідкових систем, систем автоматизованого ведення бухгалтерського обліку, експертних систем та багатьох інших програмних продуктів. Це обумовлено, перш за все, простотою та зручністю цього типу зберігання даних. 

	Система бази даних студентів пропонує користувачам уніфікований перегляд даних пов'язаних з навчанням. Щоб забезпечити єдиний узгоджений результат для кожного об’єкта, представленого в джерелах даних, об’єднання даних пов’язане з вирішенням неузгодженості даних, наявної в різнорідних джерелах даних. 

	Основною метою цього проекту є створення легкої в користуванні та надійної системи бази даних студентів, яка відстежуватиме та зберігатиме записи студентів. Ця проста у користуванні програма бази даних призначена для скорочення часу, витраченого на адміністративні завдання. Система призначена для прийняття процесу та точного створення звіту, і будь-який користувач може отримати доступ до системи в будь-який момент часу за умови доступу до Інтернету. Система також призначена для надання кращих послуг користувачам, надання значимих, узгоджених і своєчасних даних та інформації та, нарешті, підвищення ефективності шляхом перетворення паперових процесів в електронну форму.

	У цій роботі буде проаналізовано проблематику, необхідну для вирішення задачі, розглянуто спосіб реалізації даного додатку, розглянуто можливі варіанти рішень та прогнозування його майбутнього розвитку.

\chapter{Предмет дослідження в даній роботі}
\section{Опис предмету дослідження}
	Предметом дослідження – система централізованого зберігання успішності студентів, даних студентів та даних курсів навчального процесу; з можливістю отримання звітів, друку звітів у файл; з можливістю контролю діяльності викладачів. Також в програмі має бути присутня можливість форматувати звіти та таблиці: задавати різне сортування, виводити різні комбінації стовпців.

	Досліджувана система може бути частиною робочого процесу ЗВО і бути використана як система зберігання успішності та як система електронного кампусу факультетів.

\section{Поняття досліджуваного предмету}
	Успішність студента – сукупність оцінок студента за даний семестр. Оцінки розподілені між курсами та не можуть перевищувати 100 балів за курс.

	Дані студента (в контексті системи) – дані про студента, пов’язанні зі зберіганням його успішності. Ці дані включають: ПІБ, рік навчання, щифр спеціальності середній бал за поточну сесію та участь у громадській роботі.

	Курс – частина навчального процесу, серія навчальних занять з одної дисципліни. Курс може бути протяжністю в один семестр або один навчальний рік. Курс викладається деякою не пустою множиною викладачів. 

	Дані курсів (в контексті системи) – дані про курс, пов’язані зі зберіганням успішності студентів. Ці дані включають: назву курсу, семестри, в яких він проводиться, викладачів курсу, студентів що вивчають курс, РНП, та таблиці оцінок за курс.

	Звіт – результат пошуку записів в системі за певним критерієм (наприклад список всіх студентів). Звіт представляється таблицею.

	Контроль діяльності викладачів – інформація про активність викладачів у введені успішності студентів.

\section{Показники досліджувального предмету}
	Відповідно до даного визначення досліджуваного предмету, його показниками будуть:
	\begin{itemize}
		\item Комфорт перегляду успішності.
		\item Комфорт введення успішності.
		\item Повнота даних, що зберігаються.
		\item Ступінь форматування звітів та таблиць.
		\item Якість та швидкість роботи системи.
		\item Ефективність системи.
	\end{itemize}
\chapter{Дерево проблем}
	\DefaultFigure{ProblemsDiagram.jpg}{Дерево проблем}{ProblemsDiag}
\chapter{Дерево задач}
	\DefaultFigure{TaskDiagram.jpg}{Дерево задач}{TaskDiag}
\chapter{Діаграма прецедентів}
	\DefaultFigure{UseCaseDiagram.png}{Діаграма прецедентів}{UCDiag}
\chapter{Діаграма діяльності}
	\DefaultFigure{ActivityDiagram.png}{Діаграма діяльності}{ActivityDiag}
\chapter{Діаграма класів}
	\DefaultFigure{ClassDiagram.drawio.png}{Діаграма класів}{ClassDiag}
\chapter{Структура бази даних}
	\DefaultFigure{ER.drawio.png}{Структура бази даних}{ERDiag}	
\chapter{Можливі альтернативні рішення}
	В процесі проектування системи було вирішено будувати архітектуру веб-додатку за принципами трьохслойної архітектури з окремою базою даних та клієнтом. Тому альтернативами можна вважати інші види архітектури.  Введемо 3 різі альтернативи архітектури систем:
	\begin{itemize}
		\item Трьохслойна архітектура
		\item Двухслойна архітектура
		\item Мікросервіси
	\end{itemize}

	Також для оцінки альтернативних рішень будемо звертати увагу на швидкість розробки, необхідний час для доставки нової функціональності в систему при заданій архітектурі та доцільність використання. Також введему необхідну суму для обслуговування сервера(ів) для кожної архітектури.

	У випадку трьохслойної архітектури необхідно оплачувати постійний час роботи досить потужного сервера, оскільки усі операції в системі будуть виконуватись на одній машині. У випадку двухслойної сервер буде теж один але у нього має бути на порядок більше місця адже у нього не буде блоку обробки та компановки. У випадку мікросервісів, треба буде ~3-4 сервери на кожний мікросервіс. Тиждень простою нехай буде коштувати 1000\$.Розрахунок буде йти на рік роботи вперед з 3 оновленнями.

	Час на розробку на 1 місяць = 4 розробника * сер. з.п. 700\$ = 2800\$    Оцінимо 1\$ як 1 умовна одиниця. За несприятливих умов час на розробку збільшиться вдвічі.

	\begin{table}[H]
		\centering
		\begin{tabularx}{\textwidth}{ | X | X | X | X | }
			\hline
			Умова & Двухслойна & Трьохслойна & Мікросервіси\\\hline
			Розробка в місяцях & 3 & 4 & 6\\\hline
			Ціна роботи сервера за годину & 8\$ & 6\$ & 3\$ * 3.5\\\hline
			Час	на	доставку нової функціональності & 4 тижні & 2 тижні & 2 тижні\\\hline

		\end{tabularx}
	\end{table}
	Оціними за даними параметрами розробку системи, а також її підтримку протягом року, при тому, що протягом року потрібно доставити 3 нові можливості в систему:\\
	$2800*4+8*24*365+1000*2*3=87280$\\
	$2800*4*2+8*24*365+1000*2*3=98480$\\
	$2800*3+6*24*365+1000*4*3=72960$\\
	$2800*3*2+6*24*365+1000*4*3=81360$\\
	$2800*6+3*3.5*24*365+1000*2*3=114780$\\
	$2800*6*2+3*3.5*24*365+1000*2*3=131580$

	Можна побачити, що обраний варіант архітектури виявився найефективнішим з  матеріальної точки зору.\par
	\DefaultFigure{DecisionTree.png}{Дерево рішень}{DesisionTree}
\chapter{Прогрноз розвитку}
	Зараз відбувається активний процес цифровізації  практично в усіх сферах нашого життя, і навчання в ЗВО не є винятком. На заміну застарілим паперовим відомостям і журналам приходять програмні додатки, які мають значні переваги (набагато більша швидкість внесення та зчитування інформації, можливість редагування інформації, дистанційний доступ та доступ у будь-який час за потреби тощо). З кожним навчальним роком кількість університетів, які повністью переходять на цифрове збереження великої кількості інформації про студентів зростає, тому відповідно збільшується й попит на інструменти, які це можуть робити швидко та надійно. Наш додаток є саме таким інструментом.

	Сама суть роботи програмного додатку, який забезпечує зберігання та доступ до даних на віддаленому сервері є доволі простою. Тому навряд чи найближчим часом в цьому напрямку з’являться кардинально нові підходи. Натомість, розробники наразі зосереджені на вдосконаленні якості методів, які вже існують. З кожним роком зростає швидкість обробки запитів, кількість даних, яка може зберігатися, а також надійність захисту інформації. 

	Наш додаток відповідає головним критеріям якості подібних продуктів: гарантує безпеку даних, швидкий доступ до даних та зручний інтерфейс. На додаток до цього, він є доволі гнучким. За замовчуванням він має базовий функціонал, проте може буте легко розширений за вимогою замовника. Таким чином нашими потенційними клієнтами можуть бути практично будь-які заклади вищої освіти. За умови успішного старту та широкого розповсюдження нашого програмного продукту,   буде можливість також розширити область його застосування, а саме: для закладів початкової та середньої освіти. Передумовами для цього є спільна сфера інтересів (контроль за успішністю навчання) та можливість легко адаптувати додаток під вимоги шкіл, оскільки вони мають значно спрощену ієрархію у порівнянні з закладами вищої освіти. На додаток до цього можна буде надати доступ до використання нашого продукту батькам школярів, що дозволить першим якісніше контролювати процес здобування знань їхніми дітьми.

\chapter{Висновки}
	В ході виконання даної розрахунково-графічної роботи, нами було досліджено усесторонньо розглянуто та проаналізовано систему централізованого зберігання успішності студентів, даних студентів та даних курсів навчального процесу, необхідних для ефективної роботи закладу ВО.

	В ході нашого розгляду були визначені поняття та предмет досліджуваної системи, сформовано її показники та описано архітектуру. Використовуючи набуті знання, були визначені основні проблеми та їхні першопричини, що були сформовані в дерево проблем. На його основі були сформовані цілі, декомпозовані на дрібніші задачі відповідно до аналізу предметної області, які повинна виконувати досліджувана система та зображені в дереві задач.

	Наступним кроком стала побудова UML діаграм, першою з яких стала USE-CASE діаграма, що представляє собою опис того, який функціонал функціональної системи є доступним якійсь групі користувачів. Також було спроектовано діаграму діяльності (Activity diagram), яка описує вищезгаданий функціонал у вигляді бігових доріжок з кроків, кожна з яких відповідає поведінці одного з користувачів (будь то студент, або ж викладач). Дані діаграми дозволяють уявити функціональність системи, що вирішую та як нею користуватися.

	Також було розроблено діаграму класів. Для цього були визначені основні блоки з яких складається система і для кожного з них визначений виконуваний ними функціонал, структури, що вони містять, як і принципи міжблокової взаємодії. Окрім цього, оскільки наша система передбачає собою використання бази даних для збереження інформації, то була сформована «структура бази даних», яка зображує  її внутрішню архітектуру.

	% Щось про можливі альтернативні рішення

	Наостанок було здійснено аналіз можливості розвитку даної системи. В його результаті було зазначено, що даний напрям має перспективи для розвитку.

%EO document
\end{document}
