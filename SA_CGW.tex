\documentclass[14pt,a4paper,oneside]{extbook}

\usepackage{float}
\usepackage[T2A]{fontenc}
\usepackage[utf8]{inputenc}
\def\BabelCaseHack{}
\usepackage[ukrainian]{babel}
\usepackage[legalpaper, margin=2cm]{geometry}
\usepackage{graphicx}
\usepackage{fancyhdr}
\usepackage{titlesec}
\usepackage{enumitem}
\usepackage{tempora}
\usepackage{listings}
\usepackage{xcolor}
\usepackage{multirow}
\usepackage{setspace}

\pagestyle{fancy}
\renewcommand{\headrulewidth}{0pt}
\lhead{}
\chead{}
\rhead{}
\lfoot{}
\cfoot{\thepage}
\rfoot{}
\setstretch{1.6}

\titleformat{\chapter}{\normalfont\bfseries\MakeUppercase}{}{14pt}{}
\titleformat*{\section}{\normalsize\bfseries}
\titleformat*{\subsection}{\normalsize\bfseries}
\titleformat*{\subsubsection}{\normalsize\bfseries}
\titleformat*{\paragraph}{\normalsize\bfseries}
\titleformat*{\subparagraph}{\normalsize\bfseries}
\titlespacing*{\chapter}{0pt}{0pt}{0pt}
\titlespacing*{\section}{0pt}{0pt}{0pt}
\titlespacing*{\subsection}{0pt}{0pt}{0pt}
\titlespacing*{\subsubsection}{0pt}{0pt}{0pt}
\titlespacing*{\paragraph}{0pt}{0pt}{0pt}
\titlespacing*{\subparagraph}{0pt}{0pt}{0pt}

\newcommand{\LabName}{База даних обліку успішності в ЗВО}
\newcommand{\LabTheme}{Основи системного аналізу}

\newcommand{\DefaultFigure}[3]{
	\begin{figure}[H]
		\centering
		\includegraphics[width=\textwidth]{img/#1}
		\caption{#2}
		\label{pic:#3}
	\end{figure}
}

\lstset {
    commentstyle=\color{olive},
    keywordstyle=\color{red},
    numberstyle=\tiny,
    stringstyle=\color{teal},
	basicstyle={\ttfamily \scriptsize},
    breaklines=true
}

\begin{document}
\begin{titlepage}

	\begin{center}
		
		\textbf{
			МІНІСТЕРСТВО ОСВІТИ І НАУКИ УКРАЇНИ\\
			НАЦІОНАЛЬНИЙ ТЕХНІЧНИЙ УНІВЕРСИТЕТ УКРАЇНИ\\
			«КИЇВСЬКИЙ ПОЛІТЕХНІЧНИЙ ІНСТИТУТ ІМ. ІГОРЯ СІКОРСЬКОГО»\\
			ІНСТИТУТ ПРИКЛАДНОГО СИСТЕМНОГО АНАЛІЗУ\\
		}
		
		\vspace{10cm}

		Розрахунково-графічна робота\\
		з дисципліни «\LabTheme »\\
		на тему «\LabName »
		
		\vspace{3cm}

		\begin{flushright}
			Виконали:\\
			Студенти групи ДА-01\\
			ННК «ІПСА»\\
			Андрушкой Андрій, \\
			Зарицький Кирило, \\
			Колядюк Владислав, \\
			Скуратовський Артем\\
			Варіант № 3
		\end{flushright}
		
	\vfill
	Київ — 2022

	\end{center}

\end{titlepage}
\setcounter{page}{1}
\tableofcontents
%Document starts here
\chapter{Вступ}
	В сучасних закладах вищої освіти існує проблема збереження даних студентів. Студенти зацікавлені в можливості зручного перегляду оцінок за дисципліни. Викладачі зацікавлені в можливості зручного введення цих дисциплін. Адміністрація ж потребує дані про діяльність як студентів, так і викладачів.\\
	Додамо до цього пандемію 2019 року, що спричинила масове поширення вірусу, що суттєво вплинуло на освітній процес. Було введене дистанційне навчання. В його умовах постала необхідність віддаленого контролю і перегляду успішності студентів.\\
	Відповідно для успішного співпрацювання студентів та робітників ЗВО потрібна система централізованого зберігання успішності студентів.\\
	У світі практично жодна автоматизована система управління не обходиться без баз даних. БД є основою більшості сучасних довідкових систем, систем автоматизованого ведення бухгалтерського обліку, експертних систем та багатьох інших програмних продуктів. Це обумовлено, перш за все, простотою та зручністю цього типу зберігання даних. \\
	Система бази даних студентів пропонує користувачам уніфікований перегляд даних пов'язаних з навчанням. Щоб забезпечити єдиний узгоджений результат для кожного об’єкта, представленого в джерелах даних, об’єднання даних пов’язане з вирішенням неузгодженості даних, наявної в різнорідних джерелах даних. \\
	Основною метою цього проекту є створення легкої в користуванні та надійної системи бази даних студентів, яка відстежуватиме та зберігатиме записи студентів. Ця проста у користуванні програма бази даних призначена для скорочення часу, витраченого на адміністративні завдання. Система призначена для прийняття процесу та точного створення звіту, і будь-який користувач може отримати доступ до системи в будь-який момент часу за умови доступу до Інтернету. Система також призначена для надання кращих послуг користувачам, надання значимих, узгоджених і своєчасних даних та інформації та, нарешті, підвищення ефективності шляхом перетворення паперових процесів в електронну форму.\\
	У цій роботі буде проаналізовано проблематику, необхідну для вирішення задачі, розглянуто спосіб реалізації даного додатку, розглянуто можливі варіанти рішень та прогнозування його майбутнього розвитку.
\chapter{Предмет дослідження в даній роботі}
\section{Опис предмету дослідження}
	Предметом дослідження – система централізованого зберігання успішності студентів, даних студентів та даних курсів навчального процесу; з можливістю отримання звітів, друку звітів у файл; з можливістю контролю діяльності викладачів. Також в програмі має бути присутня можливість форматувати звіти та таблиці: задавати різне сортування, виводити різні комбінації стовпців.\\
	Досліджувана система може бути частиною робочого процесу ЗВО і бути використана як система зберігання успішності та як система електронного кампусу факультетів.\\
\section{Поняття досліджуваного предмету}
	Успішність студента – сукупність оцінок студента за даний семестр. Оцінки розподілені між курсами та не можуть перевищувати 100 балів за курс.\\
	Дані студента (в контексті системи) – дані про студента, пов’язанні зі зберіганням його успішності. Ці дані включають: ПІБ, рік навчання, щифр спеціальності середній бал за поточну сесію та участь у громадській роботі.\\
	Курс – частина навчального процесу, серія навчальних занять з одної дисципліни. Курс може бути протяжністю в один семестр або один навчальний рік. Курс викладається деякою не пустою множиною викладачів. \\
	Дані курсів (в контексті системи) – дані про курс, пов’язані зі зберіганням успішності студентів. Ці дані включають: назву курсу, семестри, в яких він проводиться, викладачів курсу, студентів що вивчають курс, РНП, та таблиці оцінок за курс.\\
	Звіт – результат пошуку записів в системі за певним критерієм (наприклад список всіх студентів). Звіт представляється таблицею.\\
	Контроль діяльності викладачів – інформація про активність викладачів у введені успішності студентів.\\
\section{Показники досліджувального предмету}
	Відповідно до даного визначення досліджуваного предмету, його показниками будуть:
	\begin{itemize}
		\item Комфорт перегляду успішності.
		\item Комфорт введення успішності.
		\item Повнота даних, що зберігаються.
		\item Ступінь форматування звітів та таблиць.
		\item Якість та швидкість роботи системи.
		\item Ефективність системи.
	\end{itemize}
\chapter{Дерево проблем}
	%\DefaultFigure{}{Дерево проблем}{ProblemsDiag}
\chapter{Дерево задач}
	%\DefaultFigure{}{Дерево задач}{TasksDiag}
\chapter{Діаграма прецедентів}
	\DefaultFigure{UseCaseDiagram.png}{Діаграма прецедентів}{UCDiag}
\chapter{Діаграма діяльності}
	\DefaultFigure{ActivityDiagram.png}{Діаграма діяльності}{ActivityDiag}
\chapter{Діаграма класів}
	\DefaultFigure{ClassDiagram.drawio.png}{Діаграма класів}{ClassDiag}
\chapter{Структура бази даних}
	\DefaultFigure{ER.drawio.png}{Структура бази даних}{ERDiag}	
\chapter{Можливі альтернативні рішення}

\chapter{Прогрноз розвитку}

\chapter{Висновки}
%EO document
\end{document}
